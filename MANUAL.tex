\documentclass[12pt,a4paper]{article}
\usepackage[utf8]{inputenc}
\usepackage[spanish]{babel}
\usepackage{geometry}
\usepackage{graphicx}
\usepackage{hyperref}
\usepackage{listings}
\usepackage{xcolor}
\usepackage{fancyhdr}
\usepackage{tcolorbox}
\usepackage{enumitem}
\usepackage{tikz}

\geometry{margin=2.5cm}

% Configuración de colores
\definecolor{primaryblue}{RGB}{21,101,192}
\definecolor{codegreen}{RGB}{0,128,0}
\definecolor{codegray}{RGB}{128,128,128}
\definecolor{codepurple}{RGB}{128,0,128}
\definecolor{backcolour}{RGB}{245,245,245}

% Configuración de código
\lstdefinestyle{codestyle}{
    backgroundcolor=\color{backcolour},   
    commentstyle=\color{codegreen},
    keywordstyle=\color{primaryblue},
    numberstyle=\tiny\color{codegray},
    stringstyle=\color{codepurple},
    basicstyle=\ttfamily\small,
    breakatwhitespace=false,         
    breaklines=true,                 
    captionpos=b,                    
    keepspaces=true,                 
    numbers=left,                    
    numbersep=5pt,                  
    showspaces=false,                
    showstringspaces=false,
    showtabs=false,                  
    tabsize=2
}
\lstset{style=codestyle}

% Configuración de enlaces
\hypersetup{
    colorlinks=true,
    linkcolor=primaryblue,
    filecolor=primaryblue,      
    urlcolor=primaryblue,
    pdftitle={Manual Plataforma eLearning},
    pdfpagemode=FullScreen,
}

% Encabezado y pie de página
\pagestyle{fancy}
\fancyhf{}
\fancyhead[L]{Plataforma eLearning}
\fancyhead[R]{\thepage}
\renewcommand{\headrulewidth}{0.5pt}

% Cajas informativas
\newtcolorbox{infobox}[1]{
  colback=blue!5!white,
  colframe=primaryblue,
  fonttitle=\bfseries,
  title=#1
}

\newtcolorbox{warningbox}[1]{
  colback=orange!5!white,
  colframe=orange!75!black,
  fonttitle=\bfseries,
  title=#1
}

\newtcolorbox{successbox}[1]{
  colback=green!5!white,
  colframe=green!75!black,
  fonttitle=\bfseries,
  title=#1
}

\title{
    \vspace{-2cm}
    \Huge \textbf{Manual de Usuario y Configuración}\\
    \LARGE Plataforma eLearning\\
    \vspace{0.5cm}
    \large Sistema de Gestión de Cursos en Línea
}
\author{Universidad Pontificia Bolivariana}
\date{Noviembre 2025 - Versión 1.0}

\begin{document}

\maketitle
\thispagestyle{empty}

\vspace{2cm}

\begin{center}
\begin{tcolorbox}[width=0.8\textwidth,colback=primaryblue!5!white,colframe=primaryblue,arc=5mm]
\centering
\large
Este manual documenta la configuración, instalación y uso de la plataforma eLearning, un sistema completo de gestión de cursos en línea desplegado en AWS.
\end{tcolorbox}
\end{center}

\newpage
\tableofcontents
\newpage

% ==========================================
% PARTE 1: INTRODUCCIÓN
% ==========================================

\section{Introducción}

\subsection{¿Qué es la Plataforma eLearning?}

La Plataforma eLearning es un sistema web completo para la gestión y visualización de cursos en línea. Permite a profesores crear cursos con múltiples módulos de video, mientras que los estudiantes pueden inscribirse, ver los contenidos y realizar seguimiento de su progreso.

\subsection{Características Principales}

\begin{itemize}[leftmargin=*]
    \item \textbf{Gestión de Cursos:} Los profesores pueden crear cursos y organizarlos en módulos
    \item \textbf{Videos Interactivos:} Cada módulo contiene un video educativo con seguimiento de progreso
    \item \textbf{Seguimiento Automático:} El sistema guarda automáticamente el progreso del estudiante
    \item \textbf{Roles de Usuario:} Tres tipos de usuario (Administrador, Profesor, Estudiante)
    \item \textbf{Streaming de Video:} Los videos se transmiten desde un servidor local mediante túnel seguro
    \item \textbf{Interfaz Intuitiva:} Diseño moderno con Material-UI
\end{itemize}

\subsection{Tecnologías Utilizadas}

\begin{table}[h]
\centering
\begin{tabular}{|l|l|}
\hline
\textbf{Componente} & \textbf{Tecnología} \\ \hline
Frontend & React 18.2.0 + Vite 5.0.8 \\ \hline
Backend & Node.js 18+ + Express 4.18.2 \\ \hline
Base de Datos & PostgreSQL 14 (AWS RDS) \\ \hline
Autenticación & JWT (JSON Web Tokens) \\ \hline
Servidor de Videos & Express + Cloudflare Tunnel \\ \hline
Infraestructura & AWS EC2 + RDS (us-east-2) \\ \hline
Gestión de Procesos & PM2 \\ \hline
Servidor Web & Nginx \\ \hline
\end{tabular}
\caption{Stack tecnológico de la plataforma}
\end{table}

% ==========================================
% PARTE 2: ARQUITECTURA Y CONFIGURACIÓN
% ==========================================

\section{Arquitectura del Sistema}

\subsection{Diagrama de Arquitectura}

El sistema está compuesto por cuatro componentes principales:

\begin{enumerate}
    \item \textbf{Frontend (React):} Interfaz de usuario alojada en AWS EC2
    \item \textbf{Backend (Node.js):} API REST alojada en AWS EC2
    \item \textbf{Base de Datos (PostgreSQL):} AWS RDS en región us-east-2
    \item \textbf{Servidor de Videos:} Servidor local con Cloudflare Tunnel
\end{enumerate}

\begin{infobox}{Arquitectura Cliente-Servidor}
El frontend se comunica con el backend mediante peticiones HTTP REST. El backend gestiona la lógica de negocio y se conecta a PostgreSQL para persistir los datos. Los videos se transmiten desde un servidor local a través de Cloudflare Tunnel, proporcionando acceso seguro mediante HTTPS.
\end{infobox}

\subsection{Configuración de AWS EC2}

\subsubsection{Especificaciones de la Instancia}

La plataforma está desplegada en una instancia EC2 con las siguientes características:

\begin{itemize}
    \item \textbf{Tipo de Instancia:} t2.small (2 vCPU, 2 GB RAM)
    \item \textbf{Sistema Operativo:} Ubuntu Server 22.04 LTS
    \item \textbf{Región:} us-east-2 (Ohio)
    \item \textbf{Almacenamiento:} 20 GB SSD (gp3)
    \item \textbf{IP Elástica:} 3.133.208.222
    \item \textbf{URL de Acceso:} http://3.133.208.222
\end{itemize}

\subsubsection{Grupos de Seguridad Configurados}

\begin{table}[h]
\centering
\begin{tabular}{|l|l|l|l|}
\hline
\textbf{Puerto} & \textbf{Protocolo} & \textbf{Origen} & \textbf{Propósito} \\ \hline
22 & TCP & 0.0.0.0/0 & SSH (Administración) \\ \hline
80 & TCP & 0.0.0.0/0 & HTTP (Redirección) \\ \hline
443 & TCP & 0.0.0.0/0 & HTTPS (Aplicación) \\ \hline
5000 & TCP & 0.0.0.0/0 & Backend API \\ \hline
\end{tabular}
\caption{Puertos abiertos en el firewall de EC2}
\end{table}

\subsection{Configuración de AWS RDS}

\subsubsection{Especificaciones de la Base de Datos}

\begin{itemize}
    \item \textbf{Motor:} PostgreSQL 14.x
    \item \textbf{Tipo de Instancia:} db.t3.micro (2 vCPU, 1 GB RAM)
    \item \textbf{Almacenamiento:} 20 GB SSD
    \item \textbf{Región:} us-east-2 (Ohio)
    \item \textbf{Nombre de Base de Datos:} elearning-dani-db
    \item \textbf{Puerto:} 5432
    \item \textbf{Acceso Público:} Sí (con restricciones de seguridad)
    \item \textbf{Cifrado:} SSL/TLS habilitado
\end{itemize}

\subsubsection{Esquema de Base de Datos}

La base de datos consta de las siguientes tablas principales:

\begin{lstlisting}[language=SQL, caption=Estructura de la base de datos]
-- Tabla de usuarios
CREATE TABLE users (
    id SERIAL PRIMARY KEY,
    name VARCHAR(255) NOT NULL,
    email VARCHAR(255) UNIQUE NOT NULL,
    password_hash VARCHAR(255) NOT NULL,
    role VARCHAR(20) CHECK (role IN ('admin', 'teacher', 'student')),
    created_at TIMESTAMP DEFAULT NOW()
);

-- Tabla de cursos
CREATE TABLE courses (
    id SERIAL PRIMARY KEY,
    title VARCHAR(255) NOT NULL,
    description TEXT,
    teacher_id INTEGER REFERENCES users(id),
    thumbnail_url VARCHAR(500),
    created_at TIMESTAMP DEFAULT NOW()
);

-- Tabla de módulos
CREATE TABLE modules (
    id SERIAL PRIMARY KEY,
    course_id INTEGER REFERENCES courses(id) ON DELETE CASCADE,
    title VARCHAR(255) NOT NULL,
    description TEXT,
    video_filename VARCHAR(255) NOT NULL,
    duration INTEGER DEFAULT 0,
    order_index INTEGER NOT NULL,
    created_at TIMESTAMP DEFAULT NOW()
);

-- Tabla de inscripciones
CREATE TABLE enrollments (
    id SERIAL PRIMARY KEY,
    student_id INTEGER REFERENCES users(id),
    course_id INTEGER REFERENCES courses(id),
    enrolled_at TIMESTAMP DEFAULT NOW(),
    UNIQUE(student_id, course_id)
);

-- Tabla de progreso
CREATE TABLE progress (
    id SERIAL PRIMARY KEY,
    student_id INTEGER REFERENCES users(id),
    module_id INTEGER REFERENCES modules(id),
    watched_seconds INTEGER DEFAULT 0,
    completed BOOLEAN DEFAULT FALSE,
    last_position INTEGER DEFAULT 0,
    updated_at TIMESTAMP DEFAULT NOW(),
    UNIQUE(student_id, module_id)
);
\end{lstlisting}

\subsection{Configuración del Servidor de Videos}

El servidor de videos funciona de manera local y se expone mediante Cloudflare Tunnel:

\begin{itemize}
    \item \textbf{Puerto Local:} 8080
    \item \textbf{Ruta de Videos:} C:\textbackslash Users\textbackslash danil\textbackslash Videos\textbackslash Captures
    \item \textbf{Túnel:} Cloudflare Tunnel con HTTPS
    \item \textbf{Formato de Videos:} MP4
    \item \textbf{Streaming:} Transmisión por fragmentos (chunked transfer)
\end{itemize}

\begin{infobox}{¿Por qué Cloudflare Tunnel?}
Cloudflare Tunnel permite exponer el servidor de videos local a internet de forma segura sin abrir puertos en el router. Proporciona HTTPS automático y protección DDoS, manteniendo los videos en el servidor local sin necesidad de subirlos a la nube.
\end{infobox}

\subsection{Gestión de Procesos con PM2}

El backend se ejecuta mediante PM2, un gestor de procesos para Node.js:

\begin{lstlisting}[language=bash, caption=Comandos de PM2]
# Iniciar el backend
pm2 start src/server.js --name backend

# Ver el estado
pm2 status

# Ver logs en tiempo real
pm2 logs backend

# Reiniciar el servicio
pm2 restart backend

# Detener el servicio
pm2 stop backend

# Configurar inicio automático
pm2 startup
pm2 save
\end{lstlisting}

\subsection{Configuración de Nginx}

Nginx actúa como servidor web y proxy inverso:

\begin{lstlisting}[caption=Configuración de Nginx]
server {
    listen 80;
    server_name 3.133.208.222;

    # Servir archivos estáticos del frontend
    location / {
        root /home/ubuntu/plataforma-elearning/frontend/dist;
        try_files $uri $uri/ /index.html;
    }

    # Proxy al backend
    location /api/ {
        proxy_pass http://localhost:5000;
        proxy_http_version 1.1;
        proxy_set_header Upgrade $http_upgrade;
        proxy_set_header Connection 'upgrade';
        proxy_set_header Host $host;
        proxy_cache_bypass $http_upgrade;
    }
}
\end{lstlisting}

% ==========================================
% PARTE 3: GUÍA DE USO
% ==========================================

\section{Guía de Uso por Rol}

\subsection{Acceso al Sistema}

\subsubsection{Credenciales de Prueba}

El sistema incluye tres usuarios de demostración:

\begin{table}[h]
\centering
\begin{tabular}{|l|l|l|}
\hline
\textbf{Rol} & \textbf{Email} & \textbf{Contraseña} \\ \hline
Estudiante & student@elearning.com & 123456 \\ \hline
Profesor & teacher@elearning.com & 123456 \\ \hline
Administrador & admin@elearning.com & 123456 \\ \hline
\end{tabular}
\caption{Usuarios de demostración}
\end{table}

\begin{warningbox}{Importante}
En un ambiente de producción, estas contraseñas deben ser cambiadas inmediatamente después del primer acceso.
\end{warningbox}

\subsubsection{Inicio de Sesión}

\begin{enumerate}
    \item Acceder a la URL de la plataforma: \texttt{http://3.133.208.222}
    \item Ingresar el email y contraseña
    \item Hacer clic en ``Iniciar Sesión''
    \item El sistema redirige automáticamente según el rol:
    \begin{itemize}
        \item Estudiante → \texttt{/student/cursos}
        \item Profesor → \texttt{/teacher/cursos}
        \item Administrador → \texttt{/admin/usuarios}
    \end{itemize}
\end{enumerate}

\begin{infobox}{URL de Acceso}
La plataforma está disponible públicamente en: \textbf{http://3.133.208.222}
\end{infobox}

\subsection{Rol: Administrador}

El administrador tiene control total sobre el sistema y puede gestionar usuarios.

\subsubsection{Gestionar Usuarios}

\paragraph{Ver Lista de Usuarios}

\begin{enumerate}
    \item Iniciar sesión como administrador
    \item La vista principal muestra todos los usuarios del sistema
    \item Se puede filtrar por rol usando el selector
    \item La tabla muestra: ID, Nombre, Email, Rol y Fecha de creación
\end{enumerate}

\paragraph{Crear Nuevo Usuario}

\begin{enumerate}
    \item Hacer clic en el botón ``Crear Usuario''
    \item Completar el formulario:
    \begin{itemize}
        \item \textbf{Nombre Completo:} Nombre y apellido del usuario
        \item \textbf{Email:} Dirección de correo electrónico (debe ser única)
        \item \textbf{Contraseña:} Mínimo 6 caracteres
        \item \textbf{Rol:} Seleccionar entre Admin, Profesor o Estudiante
    \end{itemize}
    \item Hacer clic en ``Crear''
    \item El sistema muestra un mensaje de confirmación
\end{enumerate}

\begin{infobox}{Nota sobre Roles}
\begin{itemize}
    \item \textbf{Admin:} Puede gestionar usuarios
    \item \textbf{Profesor:} Puede crear y gestionar cursos
    \item \textbf{Estudiante:} Puede inscribirse y ver cursos
\end{itemize}
\end{infobox}

\paragraph{Editar Usuario Existente}

\begin{enumerate}
    \item Localizar el usuario en la lista
    \item Hacer clic en el icono de editar (lápiz)
    \item Modificar los campos deseados:
    \begin{itemize}
        \item Nombre
        \item Email
        \item Rol
        \item Contraseña (opcional, dejar vacío para mantener la actual)
    \end{itemize}
    \item Hacer clic en ``Guardar Cambios''
\end{enumerate}

\paragraph{Eliminar Usuario}

\begin{enumerate}
    \item Localizar el usuario en la lista
    \item Hacer clic en el icono de eliminar (papelera)
    \item Confirmar la acción en el diálogo de confirmación
    \item El usuario es eliminado permanentemente
\end{enumerate}

\begin{warningbox}{Precaución}
Eliminar un usuario es una acción irreversible. Si el usuario es profesor, también se eliminarán todos sus cursos asociados.
\end{warningbox}

\subsection{Rol: Profesor}

Los profesores pueden crear cursos, agregar módulos y gestionar contenido educativo.

\subsubsection{Ver Mis Cursos}

\begin{enumerate}
    \item Iniciar sesión como profesor
    \item La vista principal muestra todos los cursos creados por el profesor
    \item Cada tarjeta muestra:
    \begin{itemize}
        \item Título del curso
        \item Descripción
        \item Número de módulos
        \item Botones de acción (Módulos, Editar, Eliminar)
    \end{itemize}
\end{enumerate}

\subsubsection{Crear Nuevo Curso}

\begin{enumerate}
    \item Hacer clic en el botón ``Crear Nuevo Curso'' (+)
    \item Completar el formulario:
    \begin{itemize}
        \item \textbf{Título del Curso:} Nombre descriptivo (ej: ``Introducción a JavaScript'')
        \item \textbf{Descripción:} Explicación detallada del contenido y objetivos
    \end{itemize}
    \item Hacer clic en ``Crear Curso''
    \item El sistema crea el curso y redirige automáticamente a la gestión de módulos
\end{enumerate}

\begin{successbox}{Flujo de Creación}
Después de crear un curso, el siguiente paso es agregar módulos con videos. El sistema te lleva automáticamente a la vista de gestión de módulos.
\end{successbox}

\subsubsection{Gestionar Módulos de un Curso}

\paragraph{Agregar Módulo}

\begin{enumerate}
    \item Desde la lista de cursos, hacer clic en ``Módulos''
    \item Hacer clic en el botón ``Agregar Módulo''
    \item Completar el formulario:
    \begin{itemize}
        \item \textbf{Título:} Nombre del módulo (ej: ``Variables y Tipos de Datos'')
        \item \textbf{Descripción:} Explicación del contenido del módulo
        \item \textbf{Video:} Seleccionar de la lista de videos disponibles
        \item \textbf{Orden:} Posición del módulo en la secuencia (1, 2, 3...)
    \end{itemize}
    \item Hacer clic en ``Guardar''
    \item El módulo aparece en la lista del curso
\end{enumerate}

\begin{infobox}{Ordenamiento de Módulos}
El campo ``Orden'' determina la secuencia en que los estudiantes verán los módulos. Es recomendable usar números consecutivos (1, 2, 3...) aunque el sistema acepta cualquier número entero.
\end{infobox}

\paragraph{Editar Módulo}

\begin{enumerate}
    \item En la lista de módulos, hacer clic en el icono de editar
    \item Modificar los campos deseados
    \item Hacer clic en ``Guardar Cambios''
\end{enumerate}

\paragraph{Eliminar Módulo}

\begin{enumerate}
    \item En la lista de módulos, hacer clic en el icono de eliminar
    \item Confirmar la eliminación
    \item El módulo es removido del curso
\end{enumerate}

\begin{warningbox}{Eliminar Módulos}
Al eliminar un módulo, también se elimina todo el progreso que los estudiantes hayan registrado en ese módulo.
\end{warningbox}

\paragraph{Reordenar Módulos}

Para cambiar el orden de los módulos:

\begin{enumerate}
    \item Editar el módulo que deseas mover
    \item Cambiar el valor del campo ``Orden''
    \item Guardar los cambios
    \item Los módulos se reorganizan automáticamente
\end{enumerate}

\subsubsection{Editar Curso}

\begin{enumerate}
    \item Desde la lista de cursos, hacer clic en el icono de editar
    \item Modificar el título y/o descripción
    \item Hacer clic en ``Guardar Cambios''
\end{enumerate}

\subsubsection{Eliminar Curso}

\begin{enumerate}
    \item Desde la lista de cursos, hacer clic en el icono de eliminar
    \item Confirmar la eliminación en el diálogo
    \item El curso y todos sus módulos son eliminados permanentemente
\end{enumerate}

\begin{warningbox}{Eliminación de Cursos}
Esta acción elimina:
\begin{itemize}
    \item El curso
    \item Todos los módulos asociados
    \item Todas las inscripciones de estudiantes
    \item Todo el progreso registrado
\end{itemize}
Esta acción NO puede deshacerse.
\end{warningbox}

\subsection{Rol: Estudiante}

Los estudiantes pueden explorar cursos, inscribirse y realizar seguimiento de su aprendizaje.

\subsubsection{Explorar Cursos Disponibles}

\begin{enumerate}
    \item Iniciar sesión como estudiante
    \item Hacer clic en ``Explorar Cursos'' en el menú
    \item Se muestra el catálogo completo de cursos
    \item Cada tarjeta muestra:
    \begin{itemize}
        \item Título del curso
        \item Descripción
        \item Profesor que lo imparte
        \item Número de módulos
        \item Estado de inscripción
    \end{itemize}
\end{enumerate}

\subsubsection{Inscribirse a un Curso}

\begin{enumerate}
    \item Navegar al catálogo de cursos
    \item Localizar el curso deseado
    \item Hacer clic en el botón ``Inscribirse''
    \item El sistema confirma la inscripción
    \item El curso aparece en ``Mis Cursos''
\end{enumerate}

\begin{successbox}{Inscripción Instantánea}
La inscripción es inmediata y gratuita. Una vez inscrito, tienes acceso completo a todos los módulos del curso.
\end{successbox}

\subsubsection{Ver Mis Cursos}

\begin{enumerate}
    \item Hacer clic en ``Mis Cursos'' en el menú
    \item Se muestran todos los cursos en los que estás inscrito
    \item Para cada curso se muestra:
    \begin{itemize}
        \item Título y descripción
        \item Barra de progreso general
        \item Módulos completados / Total de módulos
        \item Badge de ``Completado'' si finalizaste el curso
    \end{itemize}
\end{enumerate}

\subsubsection{Ver Contenido de un Curso}

\begin{enumerate}
    \item Desde ``Mis Cursos'', hacer clic en ``Ver Curso''
    \item La interfaz se divide en dos paneles:
    \begin{itemize}
        \item \textbf{Panel Izquierdo:} Lista de módulos con indicadores de estado
        \item \textbf{Panel Derecho:} Reproductor de video
    \end{itemize}
\end{enumerate}

\paragraph{Indicadores de Estado de Módulos}

\begin{itemize}
    \item \textbf{Círculo verde con check (✓):} Módulo completado
    \item \textbf{Círculo azul con play (▶):} Módulo actual
    \item \textbf{Círculo gris vacío (○):} Módulo no iniciado
\end{itemize}

\subsubsection{Reproducir Videos}

\begin{enumerate}
    \item Hacer clic en un módulo de la lista
    \item El video se carga automáticamente en el reproductor
    \item Usar los controles del reproductor:
    \begin{itemize}
        \item \textbf{Play/Pausa:} Iniciar o pausar el video
        \item \textbf{Barra de progreso:} Saltar a cualquier parte del video
        \item \textbf{Volumen:} Ajustar el nivel de audio
        \item \textbf{Pantalla completa:} Ver el video a pantalla completa
    \end{itemize}
\end{enumerate}

\begin{infobox}{Seguimiento Automático}
El sistema guarda automáticamente tu progreso cada 5 segundos. Puedes cerrar el navegador y al volver, el video continuará desde donde lo dejaste.
\end{infobox}

\subsubsection{Completar Módulos}

El sistema marca automáticamente un módulo como completado cuando:

\begin{enumerate}
    \item Has visto al menos el 90\% del video, O
    \item Has llegado al final del video
\end{enumerate}

Al completar un módulo:
\begin{itemize}
    \item El icono cambia a check verde (✓)
    \item La barra de progreso del curso se actualiza
    \item Puedes continuar con el siguiente módulo
\end{itemize}

\subsubsection{Completar un Curso}

Cuando completas el último módulo de un curso:

\begin{enumerate}
    \item Aparece un diálogo de felicitaciones con un trofeo dorado
    \item El mensaje indica que has completado exitosamente el curso
    \item En ``Mis Cursos'' aparece el badge de ``Completado''
    \item La barra de progreso muestra 100\%
\end{enumerate}

\begin{successbox}{¡Felicidades!}
Los cursos completados permanecen en ``Mis Cursos'' y puedes revisarlos en cualquier momento para refrescar los conocimientos.
\end{successbox}

% ==========================================
% PARTE 4: ADMINISTRACIÓN Y MANTENIMIENTO
% ==========================================

\section{Administración y Mantenimiento}

\subsection{Acceso SSH al Servidor}

Para administrar el servidor EC2:

\begin{lstlisting}[language=bash, caption=Conexión SSH]
# Desde Windows PowerShell
ssh -i "ruta/a/elearning-key.pem" ubuntu@3.133.208.222

# Ejemplo con ruta completa:
ssh -i "C:\Users\usuario\Downloads\elearning-key.pem" ubuntu@3.133.208.222
\end{lstlisting}

\subsection{Monitoreo del Sistema}

\subsubsection{Verificar Estado del Backend}

\begin{lstlisting}[language=bash, caption=Comandos de monitoreo]
# Ver estado de PM2
pm2 status

# Ver logs en tiempo real
pm2 logs backend

# Ver logs de las ultimas 100 lineas
pm2 logs backend --lines 100

# Ver informacion detallada
pm2 show backend
\end{lstlisting}

\subsubsection{Verificar Estado de Nginx}

\begin{lstlisting}[language=bash, caption=Comandos de Nginx]
# Estado del servicio
sudo systemctl status nginx

# Ver logs de error
sudo tail -f /var/log/nginx/error.log

# Ver logs de acceso
sudo tail -f /var/log/nginx/access.log

# Reiniciar Nginx
sudo systemctl restart nginx
\end{lstlisting}

\subsubsection{Verificar Base de Datos}

\begin{lstlisting}[language=bash, caption=Conexión a PostgreSQL]
# Conectar a la base de datos desde el servidor
psql -h elearning-dani-db.XXXXX.us-east-2.rds.amazonaws.com \
     -U tu-usuario \
     -d elearning-dani-db

# Ver tablas
\dt

# Ver usuarios
SELECT id, name, email, role FROM users;

# Ver cursos
SELECT id, title, teacher_id FROM courses;
\end{lstlisting}

\subsection{Respaldos de Base de Datos}

\subsubsection{Crear Respaldo Manual}

\begin{lstlisting}[language=bash, caption=Backup de PostgreSQL]
# Crear backup completo
pg_dump -h elearning-dani-db.XXXXX.us-east-2.rds.amazonaws.com \
        -U tu-usuario \
        -d elearning-dani-db \
        -F c \
        -f backup-$(date +%Y%m%d).dump

# Comprimir backup
gzip backup-$(date +%Y%m%d).dump
\end{lstlisting}

\subsubsection{Restaurar desde Respaldo}

\begin{lstlisting}[language=bash, caption=Restore de PostgreSQL]
# Restaurar backup
pg_restore -h elearning-dani-db.XXXXX.us-east-2.rds.amazonaws.com \
           -U tu-usuario \
           -d elearning-dani-db \
           backup-20251103.dump
\end{lstlisting}

\begin{warningbox}{Respaldos Automáticos}
AWS RDS realiza respaldos automáticos diarios. Se recomienda configurar una ventana de respaldo durante horas de bajo tráfico (ej: 3:00 AM).
\end{warningbox}

\subsection{Actualización del Sistema}

\subsubsection{Actualizar Backend}

\begin{lstlisting}[language=bash, caption=Actualizar código del backend]
# Conectar por SSH
ssh -i "elearning-key.pem" ubuntu@IP-EC2

# Ir al directorio del proyecto
cd /home/ubuntu/plataforma-elearning

# Obtener cambios
git pull origin main

# Instalar dependencias actualizadas
cd backend
npm install

# Reiniciar PM2
pm2 restart backend
\end{lstlisting}

\subsubsection{Actualizar Frontend}

\begin{lstlisting}[language=bash, caption=Actualizar código del frontend]
# Desde el servidor EC2
cd /home/ubuntu/plataforma-elearning/frontend

# Obtener cambios
git pull origin main

# Instalar dependencias
npm install

# Construir nueva version
npm run build

# Reiniciar Nginx
sudo systemctl restart nginx
\end{lstlisting}

\subsection{Gestión del Servidor de Videos}

\subsubsection{Agregar Nuevos Videos}

\begin{enumerate}
    \item Copiar los archivos MP4 a: \texttt{C:\textbackslash Users\textbackslash danil\textbackslash Videos\textbackslash Captures}
    \item Los videos estarán disponibles automáticamente en el dropdown de selección
    \item No es necesario reiniciar el servidor
\end{enumerate}

\subsubsection{Verificar Cloudflare Tunnel}

\begin{lstlisting}[language=bash, caption=Comandos de Cloudflare Tunnel (Windows)]
# Ver estado del servicio
cloudflared service status

# Iniciar servicio
cloudflared service start

# Detener servicio
cloudflared service stop

# Ver logs
cloudflared tail elearning-videos
\end{lstlisting}

\subsection{Solución de Problemas Comunes}

\subsubsection{Error 500 en el Backend}

\textbf{Síntomas:} Errores al cargar datos o crear recursos

\textbf{Solución:}
\begin{enumerate}
    \item Verificar logs: \texttt{pm2 logs backend}
    \item Verificar conexión a RDS
    \item Reiniciar el backend: \texttt{pm2 restart backend}
\end{enumerate}

\subsubsection{Videos No Cargan}

\textbf{Síntomas:} El reproductor muestra error al intentar reproducir

\textbf{Solución:}
\begin{enumerate}
    \item Verificar que el servidor de videos esté corriendo: \texttt{http://localhost:8080/health}
    \item Verificar estado de Cloudflare Tunnel: \texttt{cloudflared service status}
    \item Reiniciar el túnel si es necesario
    \item Verificar que el archivo de video exista en la carpeta
\end{enumerate}

\subsubsection{Frontend No Carga}

\textbf{Síntomas:} Página en blanco o error 404

\textbf{Solución:}
\begin{enumerate}
    \item Verificar estado de Nginx: \texttt{sudo systemctl status nginx}
    \item Verificar que los archivos existan: \texttt{ls /home/ubuntu/plataforma-elearning/frontend/dist}
    \item Verificar configuración de Nginx: \texttt{sudo nginx -t}
    \item Revisar logs: \texttt{sudo tail -f /var/log/nginx/error.log}
\end{enumerate}

\subsubsection{Error de Autenticación}

\textbf{Síntomas:} Error 401 al iniciar sesión

\textbf{Solución:}
\begin{enumerate}
    \item Verificar que las credenciales sean correctas
    \item Resetear contraseñas ejecutando: \texttt{node seed.js}
    \item Verificar variable de entorno \texttt{JWT\_SECRET} en el backend
\end{enumerate}

\subsection{Métricas y Rendimiento}

\subsubsection{Monitoreo de CPU y Memoria}

\begin{lstlisting}[language=bash, caption=Comandos de monitoreo]
# Ver uso de recursos
top

# Ver uso de memoria
free -h

# Ver uso de disco
df -h

# Monitoreo detallado de PM2
pm2 monit
\end{lstlisting}

\subsubsection{Optimización de Consultas SQL}

Las consultas lentas son registradas automáticamente en los logs cuando superan 1 segundo:

\begin{lstlisting}[language=bash]
# Ver consultas lentas en logs
pm2 logs backend | grep "Query lenta"
\end{lstlisting}

\subsection{Seguridad}

\subsubsection{Recomendaciones de Seguridad}

\begin{enumerate}
    \item \textbf{Cambiar contraseñas por defecto} inmediatamente en producción
    \item \textbf{Usar HTTPS} para todas las comunicaciones (configurar Let's Encrypt)
    \item \textbf{Restringir acceso SSH} solo a IPs conocidas
    \item \textbf{Actualizar dependencias} regularmente: \texttt{npm audit fix}
    \item \textbf{Configurar firewall} para limitar puertos expuestos
    \item \textbf{Habilitar autenticación de dos factores} en AWS
    \item \textbf{Rotar secretos JWT} periódicamente
    \item \textbf{Monitorear logs} en busca de actividad sospechosa
\end{enumerate}

\subsubsection{Configurar SSL con Let's Encrypt}

\begin{lstlisting}[language=bash, caption=Instalación de Certbot]
# Instalar Certbot
sudo apt install certbot python3-certbot-nginx

# Obtener certificado
sudo certbot --nginx -d tu-dominio.com

# El certificado se renueva automaticamente
# Verificar renovacion:
sudo certbot renew --dry-run
\end{lstlisting}

% ==========================================
% PARTE 5: ANEXOS
% ==========================================

\section{Anexos}

\subsection{Variables de Entorno}

\subsubsection{Backend (.env)}

\begin{lstlisting}[caption=Archivo .env del backend]
# Base de datos
DB_HOST=elearning-dani-db.XXXXX.us-east-2.rds.amazonaws.com
DB_PORT=5432
DB_NAME=elearning-dani-db
DB_USER=tu-usuario
DB_PASSWORD=tu-password-seguro

# JWT
JWT_SECRET=clave-secreta-muy-segura-cambiarla
JWT_EXPIRES_IN=7d

# Servidor
PORT=5000
NODE_ENV=production

# Video Server
VIDEO_SERVER_URL=https://videos.tudominio.com
\end{lstlisting}

\subsubsection{Frontend (.env.production)}

\begin{lstlisting}[caption=Archivo .env.production del frontend]
VITE_API_URL=http://3.133.208.222:5000
VITE_VIDEO_SERVER_URL=https://tu-tunnel.trycloudflare.com
\end{lstlisting}

\subsection{Endpoints de la API}

\subsubsection{Autenticación}

\begin{table}[h]
\centering
\small
\begin{tabular}{|l|l|p{6cm}|}
\hline
\textbf{Método} & \textbf{Endpoint} & \textbf{Descripción} \\ \hline
POST & /api/auth/register & Registrar nuevo usuario \\ \hline
POST & /api/auth/login & Iniciar sesión \\ \hline
GET & /api/auth/profile & Obtener perfil del usuario autenticado \\ \hline
\end{tabular}
\caption{Endpoints de autenticación}
\end{table}

\subsubsection{Usuarios}

\begin{table}[h]
\centering
\small
\begin{tabular}{|l|l|p{6cm}|}
\hline
\textbf{Método} & \textbf{Endpoint} & \textbf{Descripción} \\ \hline
GET & /api/users & Listar todos los usuarios (Admin) \\ \hline
POST & /api/users & Crear usuario (Admin) \\ \hline
PUT & /api/users/:id & Actualizar usuario (Admin) \\ \hline
DELETE & /api/users/:id & Eliminar usuario (Admin) \\ \hline
\end{tabular}
\caption{Endpoints de usuarios}
\end{table}

\subsubsection{Cursos}

\begin{table}[h]
\centering
\small
\begin{tabular}{|l|l|p{5.5cm}|}
\hline
\textbf{Método} & \textbf{Endpoint} & \textbf{Descripción} \\ \hline
GET & /api/courses & Listar todos los cursos \\ \hline
GET & /api/courses/:id & Obtener curso por ID \\ \hline
POST & /api/courses & Crear curso (Profesor) \\ \hline
PUT & /api/courses/:id & Actualizar curso (Profesor) \\ \hline
DELETE & /api/courses/:id & Eliminar curso (Profesor) \\ \hline
GET & /api/courses/enrolled & Cursos inscritos (Estudiante) \\ \hline
\end{tabular}
\caption{Endpoints de cursos}
\end{table}

\subsubsection{Módulos}

\begin{table}[h]
\centering
\small
\begin{tabular}{|l|l|p{5.5cm}|}
\hline
\textbf{Método} & \textbf{Endpoint} & \textbf{Descripción} \\ \hline
GET & /api/modules/course/:id & Listar módulos de un curso \\ \hline
GET & /api/modules/:id & Obtener módulo por ID \\ \hline
POST & /api/modules & Crear módulo (Profesor) \\ \hline
PUT & /api/modules/:id & Actualizar módulo (Profesor) \\ \hline
DELETE & /api/modules/:id & Eliminar módulo (Profesor) \\ \hline
\end{tabular}
\caption{Endpoints de módulos}
\end{table}

\subsubsection{Progreso}

\begin{table}[h]
\centering
\small
\begin{tabular}{|l|l|p{5cm}|}
\hline
\textbf{Método} & \textbf{Endpoint} & \textbf{Descripción} \\ \hline
GET & /api/progress/module/:id & Obtener progreso de módulo \\ \hline
POST & /api/progress & Actualizar progreso \\ \hline
POST & /api/progress/module/:id/complete & Marcar módulo completo \\ \hline
\end{tabular}
\caption{Endpoints de progreso}
\end{table}

\subsubsection{Inscripciones}

\begin{table}[h]
\centering
\small
\begin{tabular}{|l|l|p{5.5cm}|}
\hline
\textbf{Método} & \textbf{Endpoint} & \textbf{Descripción} \\ \hline
POST & /api/enrollments & Inscribirse a curso (Estudiante) \\ \hline
DELETE & /api/enrollments/:id & Cancelar inscripción \\ \hline
GET & /api/enrollments/check/:id & Verificar inscripción \\ \hline
\end{tabular}
\caption{Endpoints de inscripciones}
\end{table}

\subsection{Glosario de Términos}

\begin{description}
    \item[API REST] Interfaz de programación que permite la comunicación entre el frontend y backend mediante HTTP
    \item[AWS EC2] Servicio de computación en la nube de Amazon Web Services
    \item[AWS RDS] Servicio de base de datos relacional administrada de AWS
    \item[Backend] Servidor que procesa la lógica de negocio y gestiona los datos
    \item[Cloudflare Tunnel] Servicio que expone aplicaciones locales a internet de forma segura
    \item[Frontend] Interfaz gráfica que los usuarios ven e interactúan
    \item[JWT] JSON Web Token, método de autenticación mediante tokens
    \item[Módulo] Unidad de contenido dentro de un curso, contiene un video
    \item[Nginx] Servidor web y proxy inverso de alto rendimiento
    \item[PM2] Gestor de procesos para aplicaciones Node.js en producción
    \item[PostgreSQL] Sistema de gestión de bases de datos relacional
    \item[React] Biblioteca de JavaScript para construir interfaces de usuario
    \item[Streaming] Transmisión de video por fragmentos sin necesidad de descarga completa
\end{description}

\subsection{Acceso a la Plataforma}

\begin{infobox}{Información de Acceso}
\begin{itemize}
    \item \textbf{URL:} http://3.133.208.222
    \item \textbf{API Backend:} http://3.133.208.222:5000
    \item \textbf{Estado del Servidor:} http://3.133.208.222:5000/health
\end{itemize}
\end{infobox}

\begin{table}[h]
\centering
\begin{tabular}{|l|l|l|}
\hline
\textbf{Rol} & \textbf{Email} & \textbf{Contraseña} \\ \hline
Estudiante & student@elearning.com & 123456 \\ \hline
Profesor & teacher@elearning.com & 123456 \\ \hline
Administrador & admin@elearning.com & 123456 \\ \hline
\end{tabular}
\caption{Credenciales de acceso para evaluación}
\end{table}

\subsection{Contacto y Soporte}

Para soporte técnico o consultas sobre la plataforma:

\begin{itemize}
    \item \textbf{Universidad:} Universidad Pontificia Bolivariana
    \item \textbf{Facultad:} Ingeniería de Sistemas
    \item \textbf{Curso:} Aplicaciones con Redes
    \item \textbf{Periodo:} Sexto Semestre - 2025
\end{itemize}

\subsection{Historial de Versiones}

\begin{table}[h]
\centering
\begin{tabular}{|l|l|p{7cm}|}
\hline
\textbf{Versión} & \textbf{Fecha} & \textbf{Cambios} \\ \hline
1.0 & Nov 2025 & Versión inicial del sistema completo \\ \hline
\end{tabular}
\caption{Historial de versiones}
\end{table}

% ==========================================
% FIN DEL DOCUMENTO
% ==========================================

\newpage
\section*{Conclusión}

La Plataforma eLearning representa una solución completa y escalable para la gestión de cursos en línea. Implementada con tecnologías modernas y desplegada en infraestructura cloud de AWS, el sistema ofrece:

\begin{itemize}
    \item \textbf{Escalabilidad:} Arquitectura preparada para crecer según la demanda
    \item \textbf{Seguridad:} Autenticación JWT, cifrado SSL/TLS, y buenas prácticas de seguridad
    \item \textbf{Usabilidad:} Interfaz intuitiva con Material-UI
    \item \textbf{Rendimiento:} Streaming eficiente de videos mediante Cloudflare Tunnel
    \item \textbf{Mantenibilidad:} Código modular y bien documentado
\end{itemize}

El sistema está listo para ser utilizado en ambientes educativos reales y puede ser extendido con funcionalidades adicionales según las necesidades específicas de cada institución.

\vfill

\begin{center}
\large
\textbf{Plataforma eLearning v1.0}\\
Universidad Pontificia Bolivariana\\
Noviembre 2025
\end{center}

\end{document}
